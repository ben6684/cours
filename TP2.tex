\documentclass[11pt,twoside,a4paper]{report}
%=========================== En-Tete =================================
%--- Insertion de paquetages (optionnel) ---
\usepackage[french]{babel}   % pour dire que le texte est en français
\usepackage{a4}              % pour la taille   
\usepackage[latin1]{inputenc}     % pour les font postscript
\usepackage[T1]{fontenc}
\usepackage{mathptmx}
\usepackage{amsmath} 
\usepackage{url}
\usepackage{graphicx}
\usepackage{subfigure}
\usepackage{lmodern}
\usepackage{listings}
\usepackage{xcolor}


\definecolor{mymauve}{rgb}{0.58,0,0.82}
\definecolor{mygreen}{rgb}{0,0.6,0}

\begin{document}
\lstset{%
backgroundcolor=\color{lightgray}, 
basicstyle=\footnotesize,
commentstyle=\color{mygreen},
breaklines=true,
frame=single,
keepspaces=true,
keywordstyle=\color{blue},
language=Python,
numbers=left,
numbersep=5pt,
numberstyle=\tiny\color{black},
stringstyle=\color{mymauve}, 
xleftmargin=\parindent,
tabsize=3,
}%

\chapter{TP n�2}

\section{Probl�matique}


La mani�re de faire des requ�tes de type SQL sur des ensembles de donn�es non-structur� a �t� abord� dans le chapitre pr�c�dent et a pos� les bases de l'�criture d'algorithme MapReduce.\\

Il parait clair que les probl�matiques BigData se scindent en deux grandes cat�gories:
\begin{enumerate}
\item \underline{Technologique}, car pour faire du BigData, il faut un ensemble d'infrastructures mat�riels (Cluster, Amazon EC2, EMR, ...) et logiciels (Hadoop, NoSQL databases, ...). Qui ont pour points commun d'�tre en perpetuelle �volution et d'avoir une litt�rature plut�t bien fournie sur le net.
\item \underline{Algorithmique}, car pour pouvoir faire du BigData, il faut pouvoir int�ragir avec les donn�es dans l'ensemble des infrastructures. Pour cela, la cr�ation d'algorithmes MapReduce est n�cessaire. Des alternatives existent (Scala, Mahout, Hive ...) pour pouvoir manipuler les donn�es sans passer par l'�criture d'algorithmes MapReduce, mais ne permettent pas des traitements avanc�s et personalis�s.
\end{enumerate}

Il semble clair, qu'une �tude approfondie des possibilit�s algorithmiques MapReduce reste encore � faire. Le net dans ce cas pr�cis manque singuli�rement de litt�ratures sauf si vous compter compter des mots encore et encore.\\

Le cout de compr�hension du paradigme MapReduce est encore trop grand pour que les chercheurs et ing�nieurs aient approfondis le sujet.\\

Voil� pourquoi le cours sur MapReduce prend la place d'un simple cours sur les technologies BigData.\\

\section{Bases de statistique}

\subsection{Exercice 1}

Calcul de le prix de la vente moyenne.\\

En SQL :\\
 
\begin{lstlisting}
SELECT AVG(prix) FROM client;
\end{lstlisting}
 
En Math :\\
\begin{align}
\overline{X} = \frac{1}{n} \sum_{i=1}^{n} x_i
\end{align}

Indication : Vous devez conna�tre la somme des prix ET le nombre de vente

\subsubsection{Corrig�}

Plusieurs solutions sont possibles : \\

\begin{itemize}
\item la premi�re solution que je vois et d'envoyer en sortie du mapper la clef nulle, pour r�cup�rer sur un seul reduceur la liste des valeurs. Ainsi il suffit de sommer les valeurs et de la diviser par la taille de la liste.\\

\begin{lstlisting}
def mapper()
\end{lstlisting}

\item La deuxi�me solution est d'utiliser un combiner, 

\item La troisi�me est d'enchainer deux job map-rduce

\end{itemize}


\end{document}
